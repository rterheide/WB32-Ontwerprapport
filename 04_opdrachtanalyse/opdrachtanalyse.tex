\chapter{Opdrachtanalyse}
\label{cha:opdrachtanalyse}


\textit{In dit hoofdstuk wordt de ontwerpuitdaging geanalyseerd. Hierdoor wordt duidelijk wat er exact verwacht wordt van het eindontwerp en hoe dit gerealiseerd kan worden. De ontwerpuitdagingen worden in  \cref{se:Ontwerp_uitdagingen} opgesteld. In \cref{se:functieanalyse} is de functieanalyse te vinden. De ingewonnen informatie is te vinden in \cref{se:Ingewonnen_informatie} en \cref{cha:bijlage_C}. Het programma van eisen is te vinden in \cref{se:PVE}. In \cref{se:PC} zijn de prestatie criteria te vinden.}

\section{Ontwerp uitdagingen}
\label{se:Ontwerp_uitdagingen}
De ontwerpuitdagingen van het pakkethondje zijn:

\begin{enumerate}
    \item Er moet een betrouwbare aandrijving en energiebron gekozen worden.
    \item Er moet een mechanisme ontworpen worden voor het nemen van de hindernissen op het parcours.
    \item Er moet een manier bedacht worden om het pakket te ontvangen en af te leveren.
    \item Er moet een manier gevonden worden om het pakket te vervoeren, zonder dat het van het pakket hondje valt.
    \item Het complete Pakket hondje moet te produceren zijn in vijf werkplaats momenten van vier uur, passend binnen een budget van €200/€250.
\end{enumerate}



\section{Functieanalyse}
\label{se:functieanalyse}
De functies van het pakket hondje kan ingedeeld worden in hoofd en deelfuncties.\\
\vspace{\baselineskip}

De hoofdfuncties van het hondje zijn:
\begin{enumerate}
    \item Het mechanisme moet een lading van 10kg van A naar B kunnen verplaatsen.
    \item Het mechanisme moet over twee obstakels kunnen bewegen.
\end{enumerate}
\vspace{\baselineskip}
De deelfuncties van het hondje zijn:
\begin{enumerate}
    \item Het pakket hondje moet een interne energie bron hebben die de aandrijving verzorgt.
    \item Het pakket hondje moet een stuurmechanisme hebben.
    \item Het pakket hondje moet bestuurbaar kunnen zijn.
    \item Het pakket hondje moet het pakket ontvangen op een gewenste hoogte.
    \item Het pakket hondje moet in staat zijn om gedurende de hindernis baan het pakketje bij zich te houden.
    \item Het pakket hondje moet in staat zijn om te kunnen remmen wanner dat gewenst is.
    \item Het pakket hondje moet in staat zijn om voort te kunnen bewegen.
\end{enumerate}


\section{Ingewonnen informatie}
\label{se:Ingewonnen_informatie}
Voor het begin van het ontwerp proces is het van belang om zoveel mogelijk informatie in te winnen dat een breder beeld geeft over de mogelijke opties voor de ontwerpopdracht. De ingewonnen informatie is te zien in \cref{cha:bijlage_C}. Hieronder wordt een korte samenvatting gegeven over de onderwerpen die zijn onderzocht en welke conclusie daaruit is getrokken:\\

\begin{itemize}
    \item Mogelijke kosten
    \item Bestaande mechanismes die dezelfde functie verrichten.
    \item Mogelijke stuur mechanismes
\end{itemize}
\vspace{\baselineskip}

\textbf{Mogelijke kosten.} Uit het onderzoek naar prijzen van onderdelen is gebleken dat elektra een grote kostenpost is. Dit heeft te maken met de relatief dure prijs van elektromotoren en accu's, (\cite{123accu.nl} en \cite{Conrad}).
\vspace{\baselineskip}

\textbf{Bestaande mechanismes die dezelfde functie verrichten.} Een mogelijk interessante machine is een mars lander ( \cite{r.g._bonitz_nguyen_kim_bonitz_folkner_golombek_olson_spohn_grott_et_al._1970}).
\vspace{\baselineskip}

\textbf{Mogelijke stuur mechanismes.} Bij het onderzoek naar stuur mechanismes naast de al bekende manieren van sturen, zoals een fiets, ook alternatieve manieren onderzocht, zie \cite{Stuurmechanisme}.
\vspace{\baselineskip}


\vspace{\baselineskip}

\section{Programma van eisen}
\label{se:PVE}

De belangrijkste functionele eisen van de ontwerpopdracht zijn: 
\begin{enumerate}
    \item Het pakket hondje mag tijdens het parcours niet zijn stabiliteit verliezen of omvallen.
    \item Het pakket hondje moet zodanig bestuurd kunnen worden dat het kan worden aangedreven, afgeremd en kan sturen.
    \item Het pakket hondje moet in (eventueel gedemonteerd) in een verhuisdoos passen van $ 0,48 \times 0,32 \times 0,33 m.$
    \item Het mechanisme moet niet bezwijken bij een toegevoegde belasting van 10 kg 
    \item Bij de constructie van het pakket hondje mogen alleen frame onderdelen gelast worden.
    \item Het pakket hondje moet zijn eigen energiebron vervoeren.
    \item Het pakket hondje moet conform de CE-norm zijn.
    \item Het pakket hondje mag tijdens het parcours het pakket niet verliezen.
    \item Het pakket hondje moet in maximaal 10 minuten de hindernisbaan afleggen.
    \item De trekkracht aan het geleidende koord moet op elk moment minder dan 25 newton zijn.
\end{enumerate} 
\vspace{\baselineskip}

\section{Prestatie criteria}
\label{se:PC}

De prestatie criteria van het pakket hondje zijn:
\begin{enumerate}
    \item Het pakket hondje moet een zo moeilijk mogelijk obstakel kunnen nemen.
    \item Het pakket hondje moet zo goed mogelijk statisch en dynamisch te analyseren zijn.
    \item Het pakket hondje moet zo goedkoop mogelijk zijn.
    \item Het ontwerp moet zo min mogelijk gelaste onderdelen en zo veel mogelijk makkelijk vervangbare onderdelen bevatten (duurzaamheid).
    \item Het pakket hondje moet makkelijk te construeren en assembleren zijn.
    \item Het pakket hondje moet zo licht mogelijk zijn.
    \item Het pakket hondje moet zo min mogelijk bewegende onderdelen bevatten.
    \item Het pakket hondje moet zo snel mogelijk zijn.
    \item Het pakket hondje moet zo wendbaar mogelijk zijn.
\end{enumerate}

