\chapter{Inleiding}
\label{cha:inleiding}
In 2018 is het globale pakketvolume met 16\% gestegen, van 74,7 miljard naar 87 miljard pakketjes (\cite{pitney_bowes_2018}) Men verwacht dat het aantal pakketjes 100 miljard zal bereiken aan het eind van 2020. De groei in deze markt zorgt voor een toenemende werklast van de bezorgers. Zij hebben ook een hoger risico op schouder- en rugpijn dan werknemers in andere branches. (\cite{hurley_marshall_hogan_wells_2012}). Het implementeren van een 'pakkethondje' kan helpen om zulke medische klachten te voorkomen. Eerstejaars werktuigbouwkundestudenten aan de TU Delft hebben de uitdaging gekregen om zo'n pakkethondje te ontwerpen en ontwikkelen (\cite{beek_2020_opdracht}).
\vspace{\baselineskip}

Het doel van dit rapport is het ontwerpproces van het pakkethondje in kaart te brengen en een voorspelling van de prestatie te maken. De opdracht die door TU Delft is samengesteld luidt: “De uitdaging is om een “pakkethondje” te ontwerpen dat een pakket met een gewicht van iets minder dan 10kg foutloos over een hindernisbaan weet te transporteren.” (\cite{beek_2020_opdracht}). Tijdens het ontwerpproces is er specifiek rekening gehouden met de wens dat het pakkethondje afstandbestuurbaar is. Het project kan worden opgedeeld in drie fases: het ontwerpproces, de vervaardiging en de prestatie test.
\vspace{\baselineskip}

De opbouw van dit rapport is als volgt: in \cref{cha:opdrachtanalyse} word de opdrachtanalyse, hierbij is rekening gehouden met de prestatiecriteria om een goed geïnformeerd programma van eisen op te stellen. Aan de hand van het PvE zijn er deeloplossingen en spuugmodellen gemaakt die zijn uitgewerkt tot totaaloplossingen wat in \cref{cha:ideeOntwikkeling} wordt uitgewerkt. De evaluatie van de totaaloplossingen wordt in \cref{cha:Concept_keuze} behandeld, met de evaluatie is er een keuze gemaakt voor het eindontwerp op basis van de vastgestelde prestatiecriteria. In \cref{cha:gekozenConcept} wordt het eindontwerp volledig tot detail uitgewerkt met de faal- en prestatieanalyse. Vervolgens wordt in \cref{cha:gekozenConcept} een conclusie geschreven op het ontwerpproces. 
