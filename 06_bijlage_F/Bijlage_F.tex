\chapter{Python scripts}
\label{Cha: Bijlage_F}

\section{wielassen.}


\begin{lstlisting}[language=Python][style=mystyle]

import numpy as np
import matplotlib.pyplot as plt

#Constanten:

breedte_PH = 0.40               #m, breedte van het PH
l = breedte_PH                  #m, lengte van de as (ashouder tot ashouder)
a = 0.002                       #m, dikte as houders
b = l - 2*a                     #m, ashouder tot ashouder
c = 0.005                       #m, afstand van wiel tot ashouder
m_totaal = 23                   #kg, totale massa van het pakket hondje

g = 9.81                        #m/s^2, gravitatie versnelling
verdeling_gewicht = (1/3)       #Verdeling van het gewicht op een as
d_as = 0.008                    #m, diameter van een as
r_as = (1/2) * d_as             #m, straal van een as
A_as = np.pi * r_as**2,         #m^2, oppervlak van de doorsnede van de as

c = r_as                        #m, grootste vezel afstand
I = (np.pi / 64) * d_as**4      #m^4, traagheidsmoment

E_stl_A36 = 200 * 10**9         #Pa
Yield_strength = 376 * 10**6    #Pa     

x_lijst = np.linspace(0, 2*c + l, 1000)         #Lege lijst voor de positie
Q = np.zeros(len(x_lijst))                      #lege lijs van de dwarskrachten
M = np.zeros(len(x_lijst))                      #lege lijst voor de momenten in de as


#Krachten analyse:
Fz = m_totaal * g                               #N, zwaartekracht van het pakkethondje
F_totaal = (1/2) * Fz                           #N, kracht die het gewicht uitoefent op een as

F = F_totaal * verdeling_gewicht            #kracht op as door de ashouder bij de eerste ashouder, Newton
P = F_totaal * (1 - verdeling_gewicht)      #kracht op as door de ashouder bij de tweede ashouder, Newton

S = ( (c + (1/2)*a)*F + (c + (3/2)*a + b)*P ) / (2*c + l)       #reactiekracht bij het eerste wiel , Newton
R = F + P - S                                                   #reactiekracht bij het tweede wiel, Newton

#print stuk:
print (' ')
print('De totale aangenomen massa =', m_totaal, 'kg')
print('De aangenomen gewichtsverdeling op de twee ashouders =', np.round(verdeling_gewicht, 3), 'tot', np.round(1 - verdeling_gewicht, 3), 'respectievelijk')
print(' ')
print('Hieruit volgt:')
print('De kracht in de eerste ashouder =', np.round(F,2), 'N')
print('De kracht in de andere ashouder =', np.round(S,2), 'N')

#formules voor de krachten en moment lijnen in de as:

#voor het eerste deel van de as ( 0 < x < c + (1/2)a), x1:
x1_max = c + (1/2)*a        #Afstand van het eerste wiel tot aangrijpingspunt van de kracht bij de eerste ashouder

def Q1(x):
    Q = R
    return Q

def M1(x):
    M = -x*R
    return M

#voor het tweede deel van de as ( c+(1/2)a < x < c + b + (3/2)a), x2:
x2_max = c + b + (3/2)*a    #Afstand van het eerste wiel tot het aangrijpingspunt van de kracht bij de tweede ashouder

def Q2(x):
    Q = R - F
    return Q

def M2(x):
    M = ( x - c - (1/2)*a )*F - x*R
    return M

#voor het derde deel van de as ( c + b + (3/2)a) < x < l + 2*c):
x3_max = l + 2*c            #Afstand van wiel tot wiel

def Q3(x):
    Q = R - P - F
    return Q

def M3(x):
    M = (x - c - b - (3/2)*a)*P + (x - c - (1/2)*a)*F - x*R
    return M

#For loop:
for i in range(len(x_lijst)):
    x = x_lijst[i]
    if x < x1_max:
        M[i] = M1(x)
        Q[i] = Q1(x)
    elif x < x2_max:
        M[i] = M2(x)
        Q[i] = Q2(x)
    elif x < x3_max:
        M[i] = M3(x)
        Q[i] = Q3(x)
      
    
    
#plotten en script:
plt.plot(x_lijst, M, label = 'momenten lijn')
plt.xlabel('afstand (m)')
plt.ylabel('moment (N/m)')
plt.grid()
plt.legend()

plt.figure()
plt.plot(x_lijst, Q, label = 'dwarskrachten lijn')
plt.xlabel('afstand (m)')
plt.ylabel('dwarskracht (N)')
plt.grid()
plt.legend()

print(' ')
M_max = min(M)    
print('Het maximale buigmoment in de as =',np.abs(np.round(M_max,2)), 'N/m')  
Q_max = max(Q)
print('De maximale afschuifkracht in de as =', np.abs(np.round(Q_max, 2)), 'N')
print(' ')

shear_stress_max = 1.4 * (Q_max / A_as)         #N, maximale dwarskracht die op as wordt uitgeoefend
shear_stress_allow = Yield_strength             #N, yield strength van staal A36

sigma_max = (M_max * c) / I                     #N, maximale drukkracht die op de as wordt uitgeoefend
sigma_allow = Yield_strength                    #N, yield strength van staal A36

if shear_stress_max < shear_stress_allow and sigma_max < sigma_allow:
    print('De stalen as zou het gewicht kunnen houden')
else:
    print('De stalen as zou het gewicht niet kunnen houden')

\end{lstlisting}

\section{Schaarliften}
\label{se: bijlage_F schaarliften}
\vspace{\baselineskip}

\begin{lstlisting}[language=Python][style=mystyle]
import numpy as np

#constanten:

h1 = 0.45                    #meter, hoogte van ontvangen pakket
h2 = 0.15                   #meter, hoogte tijdens het rijden
h3 = 0.55                   #meter, hoogte tijdens het nemen van het talud
Theta_max = 50              #graden, maximale hoek
Breedte_toelaatbaar = 0.4   #meter, breedte van de wielbasis
m_totaal = 18                       #kg
g = 9.81                            #m/s^2
b_member = 0.005                    #m
d_member = 0.01                     #m
E_stl_A36 = 200 * 10 ** 9           #Pa
E_stl_A992 = 200 * 10 ** 9          #Pa
E_stl_304 = 193 * 10 ** 9           #Pa
E_alu = 73.1 * 10 ** 9              #Pa
l_verhuisdoos = 0.48                #m
t_dikte_koker = 0.002               #m
b_breedte_koker = 0.035              #m
h_hoogte_koker = 0.035              #m

#berekeningen, lengtes en hoeken

Theta_max_rad = (Theta_max  * 2 * np.pi) / 360

l = (0.5 * h3) / ( np.sin(Theta_max_rad) )

Theta_min_rad = np.arcsin( (0.5 * h2) / l)

Theta_min = (Theta_min_rad*360) / (2*np.pi)

Breedte = ( l**2 - (0.25 * h2**2) )**0.5


#berekeningen, krachten en richtingen

Theta_member_rad = Theta_min_rad    #rad

F_member = (m_totaal * g) / np.sin(Theta_member_rad)
F_draad = F_member * np.cos(Theta_member_rad)

opp_member = b_member * d_member
sigma_member = F_member / opp_member

#berekeningen knik:

I_rechthoek = (1/12) * b_member * d_member**3
I_koker = (1/12)*b_breedte_koker*(h_hoogte_koker)**3 - ((1/12)*(b_breedte_koker-2*t_dikte_koker)*(h_hoogte_koker-2*t_dikte_koker)**3)

#scripts

print(' ')
print('Stel:')
print(' ')
print('m_totaal =', m_totaal, 'kg')
print('Theta_max =', Theta_max, 'graden')
print('Maximale toelaatbare breedte =', Breedte_toelaatbaar, 'm')
print(' ')
print('De hoogte van het chassis als het pakkethondje rijdt,            h2, is:', h2, 'm')
print('De hoogte die het meubelhondje heeft voor pakket laden/lossen,       h1, is:', h1, 'm')
print('De hoogte die het pakkethondje heeft tijdens het talud nemen,        h3, is:', h3, 'm')
print(' ')
print('dan geld het volgende,')
print(' ')
print(' ')
print(' ')

print('DE WAARDEN VOOR DE SCHAARLIFTEN ZIJN:')
print(' ')
print('THETA_MAX:', Theta_max, ' graden')
print('THETA_MIN:', np.round(Theta_min, 2), ' graden')
print('BREEDTE MEMBER:', np.round(Breedte, 2), ' meter')
print('LENGTE MEMBER:', np.round(l, 2), ' meter')
print(' ')
if l >= l_verhuisdoos:
    print('De members passen niet in een verhuisdoos')
else:
    print('De members passen in een verhuisdoos')
print(' ')
print(' ')
print('VOOR DE KRACHTEN:')
print(' ')
print('KRACHT IN MEMBER:', np.round(F_member,2), 'N', '. Het member staat in druk.')
print('KRACHT IN DRAAD:', np.round(F_draad,2),'N')
print('SPANNING IN MEMBER:', np.round(sigma_member, 2), 'N/m')
print(' ')

if Breedte < Breedte_toelaatbaar:
    print('De maximale breedte die de schaarlift aanneemt is toegestaan.')
else:
    print('De maximale breedte die de schaarlift aanneemt is niet toegestaan')
    
print(' ')
print(' ')
print('VOOR DE DRUKBELASTING GELDT:')
print(' ')



E_metalen = [E_stl_A36, E_stl_A992, E_stl_304, E_alu]
metalen= ['stl_A36', 'stl_A992', 'stl_304', 'aluminium']

print('Voor een rechthoek profiel geldt:')
print(' ')
for i in range(len(E_metalen)):
    E = E_metalen[i]
    F_knik = (2.046*(np.pi)**2 * E * I_rechthoek ) / (((1/2)*l)**2)
    print(np.round(F_knik, 2), 'N ,is de maximale druk kracht voor', metalen[i])
    if F_knik > F_member:
        print('De Schaarlift kan de drukkracht aan.')
    else:
        print('De schaarlift kan de drukkracht niet aan.')
    print(' ')

print(' ')
print(' ')
print('Voor een koker profiel geldt:')
print(' ')
for i in range(len(E_metalen)):
    E = E_metalen[i]
    F_knik = (2.046*(np.pi)**2 * E * I_koker ) / (((1/2)*l)**2)
    print(np.round(F_knik, 2), 'N ,is de maximale druk kracht voor', metalen[i])
    if F_knik > F_member:
        print('De Schaarlift kan de drukkracht aan.')
    else:
        print('De schaarlift kan de drukkracht niet aan.')
    print(' ')

\end{lstlisting}
\vspace{\baselineskip}

\section{U-profiel aan de onderkant van de kar}
\label{se: onderkant_u-profiel}

\begin{lstlisting}[language=Python][style=mystyle]


import numpy as np
import matplotlib.pyplot as plt

#constanten:

lengte_member = 0.36    #m, lengte van een link
l = 0.38            #m, breedte pakkethondje
a = 0.05            #m, afstand van eerste links tot ashouder
Theta_max = 50      #graden, maximale hoek van de schaarliften
Theta_max_rad = (Theta_max*np.pi*2) /360        #rad, maximale hoek van de schaarliften
m_totaal = 20                                   #kg
verdeling_gewicht = (2/3)                       #gewicht op een schaarlift (maximaal)
verdeling_kracht = (1/2)
g = 9.81                                        #m/s^2
dist = np.cos(Theta_max_rad) * lengte_member    #m, afstand van de twee links op het u-profiel
z = l - a - dist

d_profiel = 0.002
l_vlap = 0.01
h_profiel = 0.01
c = 0.5 * h_profiel

A_profiel = 2 * (d_profiel * l_vlap) + (h_profiel * d_profiel)

d1 = a                                  
d2 = a + dist

x_lijst = np.linspace(0,l,1000)
Q = np.zeros(len(x_lijst))
M = np.zeros(len(x_lijst))

E_stl_A36 = 200 * 10**9         #Pa
Yield_strength_stl = 376 * 10**6    #Pa   (staal)

E_alu = 70 * 10**9              #Pa
Yield_strength_alu = 109 * 10**6 #Pa

#voor het traagheidsmoment geldt:

I_1 = (1/12)*d_profiel*h_profiel**3
I_3 = (1/12)*l_vlap*d_profiel**3 + (l_vlap*d_profiel)*( 0.5*h_profiel )**2

I_koker = 2*I_1 + I_3


#Ontbinding van de krachten
Fz = m_totaal * g                               #N
Fz_deel = Fz * verdeling_gewicht                #N
Fx = Fz_deel * np.cos(Theta_max_rad)            #N, kracht in de horizontale richting uitgeoefend op het profiel 
Fy = Fz_deel * np.sin(Theta_max_rad)            #N, kracht in de verticale richting uitgeoefend op het profiel
F_member_1 = Fy * verdeling_kracht              #N, Kracht dat het eerste member uitoefent op het profiel
F_member_2 = Fy * (1-verdeling_kracht)          #N, Kracht dat het tweede member uitoefent op het profiel

R_2 = ( a*F_member_1 + (a+dist)*F_member_2) / l #N, reactiekracht bij de eerste ashouder
R_1 = F_member_1 + F_member_2 - R_2             #N, reactiekracht bij de tweede ashouder

#voor het eerste deel van het profiel ( 0 < x < d1), x1:
x1_max = d1

def Q1(x):
    Q = R_1
    return Q

def M1(x):
    M = -x*R_1
    return M

#voor het tweede deel van de as ( x1_max < x < x2_max), x2:
x2_max = a + dist

def Q2(x):
    Q = R_1 - F_member_1
    return Q

def M2(x):
    M = ( x - a )*F_member_1 - x*R_1
    return M

#voor het derde deel van de as ( x2_max < x < x3_max):
x3_max = l 

def Q3(x):
    Q = R_1 - F_member_1 - F_member_2
    return Q

def M3(x):
    M = (x - a - dist)*F_member_2 + (x - a)*F_member_1 - x*R_1
    return M

for i in range(len(x_lijst)):
    x = x_lijst[i]
    if x < x1_max:
        M[i] = M1(x)
        Q[i] = Q1(x)
    elif x < x2_max:
        M[i] = M2(x)
        Q[i] = Q2(x)
    elif x < x3_max:
        M[i] = M3(x)
        Q[i] = Q3(x)

#plotten en script:

plt.plot(x_lijst, M, label = 'momenten lijn')
plt.xlabel('afstand (m)')
plt.ylabel('moment (N/m)')
plt.grid()
plt.legend()


plt.figure()
plt.plot(x_lijst, Q, label = 'dwarskrachten lijn')
plt.xlabel('afstand (m)')
plt.ylabel('dwarskracht (N)')
plt.grid()
plt.legend()

print(' ')
M_max = min(M)    
print('Het maximale buigmoment in de as =',np.abs(np.round(M_max,2)), 'N/m')  
Q_max = max(Q)
print('De maximale afschuifkracht in de as =', np.abs(np.round(Q_max, 2)), 'N')
print(' ')

sigma = Fx / A_profiel
    
shear_stress_max = 1.4 * (Q_max / A_profiel)
shear_stress_allow_stl = Yield_strength_stl
shear_stress_allow_alu = Yield_strength_alu

sigma_max = (M_max * c) / I_koker
sigma_allow_stl = Yield_strength_stl
sigma_allow_alu = Yield_strength_alu

if sigma_max < sigma_allow_stl and sigma < Yield_strength_stl:
    print('Het stalen profiel is in staat op de kracht van het pakkethondje op te vangen')
else:
    print('Het stalen profiel is niet in staat op de kracht van het pakkethondje op te vangen')

if sigma_max < sigma_allow_alu and sigma < Yield_strength_alu:
    print('Het aluminium profiel is in staat op de kracht van het pakkethondje op te vangen')
else:
    print('Het aluminium profiel is niet in staat op de kracht van het pakkethondje op te vangen')

print(' ')
print('De reactie kracht van het eerste wiel op de aarde =', np.round(R_1,2), 'N')
print('De reactie kracht van het tweede wiel op de aarde =', np.round(R_2,2), 'N')


\end{lstlisting}





