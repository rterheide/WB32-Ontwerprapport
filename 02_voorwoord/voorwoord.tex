\chapter*{Voorwoord}
\label{cha:voorwoord}
\addcontentsline{toc}{chapter}{Voorwoord}
\setheader{Voorwoord}

Het eerste jaar van de studie Werktuigbouwkunde aan de TU Delft wordt afgesloten met de  Eerstejaars Ontwerpwedstrijd. Tijdens het project dat hieraan verbonden is krijgt een groep studenten een opdracht, welke elk jaar verschilt, en moeten hier samen voor een half jaar aan werken om tot een zo goed mogelijk resultaat te komen. 

Dit jaar is de opdracht van de ontwerpwedstrijd om een ‘pakkethondje’ te bouwen, afgeleid van het welbekende meubelhondje. Het pakkethondje zou gebruikt moeten worden om het afleveren van pakketjes tot aan de deur makkelijker te maken. Wij, bestaande uit 6 eerstejaars studenten Werktuigbouwkunde, zijn een van vele groepjes die dit jaar meedoen met de ontwerpwedstrijd. In dit rapport is het proces besproken hoe dit project is aangepakt en tot welk ontwerp wij uiteindelijk zijn gekomen.

Tijdens het ontwerpproces hebben wij ons op verschillende manieren laten inspireren. We hebben gekeken naar hoe bepaalde dieren zich in de natuur voortbewegen over obstakels en nagedacht of dit ook in dit project zou kunnen worden toegepast. Ook hebben wij ACCREx toegepast en een morfologische kaart gebruikt om tot zoveel mogelijk deeloplossingen te komen voor de verschillende ontwerp uitdagingen.

In \cref{cha:Concept_keuze} word voor de geïnteresseerde uitgelicht hoe de verschillende deelconcepten hebben geleid tot een uiteindelijk definitief concept. Wie juist geïnteresseerd is in hoe we tot de deeloplossingen zijn gekomen voor de verschillende ontwerp uitdagingen kunnen wij verwijzen naar \cref{cha:ideeOntwikkeling}, waar de idee ontwikkeling in wordt uitgewerkt.

Tot slot willen wij nog een aantal personen bedanken van wie wij de nodige hulp hebben gehad om tot het eindproduct te komen: onze project docent Wim van Son, onze projectmentor Christopher Combes, onze Schriftelijk Rapporteren docent Regina Tange-Hoffmann en de docent van het vak WOP 3A Anton van Beek.

\vspace{\baselineskip}
Delft, \today\\
\begin{table}[h]
    \begin{tabular}{l}
        Hilbert Bijzitter\\
        Robin ter Heide\\
        Max Leenstra\\
        Kevin Tran\\
        Floris Kuck\\
        Beer Mook
    \end{tabular}
\end{table}

\vspace{\baselineskip}


