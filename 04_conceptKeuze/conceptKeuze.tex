\chapter{Concept Keuze}
\label{cha:Concept_keuze}

\textit{ In dit hoofdstuk worden alle concepten uit \cref{cha:ideeOntwikkeling} geanalyseerd en wordt het meest kansrijke eind concept aangewezen. Dit gebeurt op basis van het programma van eisen gegeven in \cref{se:PVE}. In \cref{se:Gewogen_criteria_methode} wordt de gewogen criteria methode toegelicht en toegepast. Dit is onderverdeeld in een stuk over de weging van de criteria (\cref{se:weging_criteria}) en een stuk toelichting over de gegeven scores (\cref{se:gegeven_scores}). Tot slot kan in \cref{se:keuze_van_concept} gevonden worden welk concept is gekozen en waarom.}

\section{Gewogen criteria methode}
\label{se:Gewogen_criteria_methode}
Uit het onderzoek van de gewogen criteria methode heeft de inklapper de hoogste score gekregen. Het maken van een concept keuze is gebeurt door middel van de gewogen criteria methode, \cite{hanus_hagger_proffitt_barnes}.

Deze methode begint met het afwegen van de gekozen prestatie criteria. Hoe belangrijk zijn de criteria ten opzichte van elkaar. Daarna krijgen de concepten scores die worden vermenigvuldigd met de weging die criteria hebben gekregen. Wat hier dan uitkomt is een ranking van verwachte prestaties van de verschillende concepten.\\


\subsection{Weging criteria}
\label{se:weging_criteria}
Voor het uitvoeren van de gewogen criteria methode is het van belang om de prestatie criteria te rangschikken op belangrijkheid. Deze rangschikking is te vinden in \cref{ch:gewogen_criteria_methode}. \\
Tijdens de afweging welke criteria belangrijker worden gevonden dan anderen, zijn er ook verschillende dilemma's naar boven gekomen. Dit zijn een paar criteria die het bespreken waard zijn:  \\

\vspace{\baselineskip}
\textbf{Wendbaarheid.} Hierin is te zien dat er een lage ranking gegeven is aan wendbaarheid. Dit komt omdat de criteria die zijn opgesteld, afgewogen zijn op de reglementen die waren gegeven aan de opdracht (\cite{beek_2020_wedstrijdregelement}). Hierdoor zijn de criteria afgewogen op hoe ze de prestatie zouden kunnen verminderen. Wendbaarheid is minder van belang omdat het meubel hondje maar twee keer 90 graden hoeft te draaien en moet kunnen bijsturen. De tijd die het concept daarvoor nodig heeft moet snel genoeg zijn maar is niet essentieel voor de prestatie van het concept.\\
\vspace{\baselineskip}

\textbf{Voorspelbaarheid.} Uit de gewogen criteria methode is naar boven gekomen dat het belangrijk is hoe voorspelbaar het meubel hondje is en hoe moeilijk het obstakel is dat hij kan nemen. De voorspelbaarheid is hoog geëindigd omdat het van belang is om te kunnen weten wat er te wachten staat. Als het ontwerp voldoende voorspelbaar is moet het ontwerp in principe voldoende kunnen worden ontworpen.\\

\textbf{Zo makkelijk mogelijk te assembleren / construeren.} Wat ook een relatief hoge score heeft gekregen is hoe makkelijk mogelijk het concept te assembleren/construeren is. Dit heeft een hoge score gekregen omdat het van belang is om een zo eenvoudig mogelijk ontwerp te krijgen. Als het concept eenvoudig wordt, is het mogelijk om het ontwerp perfect te laten functioneren. \\
\vspace{\baselineskip}



\subsection{Gegeven scores}
\label{se:gegeven_scores}
Om een concept te kiezen is het van belang om de concepten te vergelijken op de verschillende criteria. Voor de gegeven scores en het totaal aantal punten dat eruit is gekomen zie \cref{ch:gewogen_criteria_methode}. Uit deze resultaten is dat de volgende ranking eruit is gekomen: \\

\begin{center}
 \begin{tabular}{| l | c | r | }
  \hline			
  score: & concept: & punten:\\
  1 & Inklapper & 184 \\
  2 & Vierpoter & 180 \\
  3 & Mantis Car & 174 \\
  4 & Uitschuiver & 165 \\
  \hline  
 \end{tabular}
\end{center}
\vspace{\baselineskip}

Hieronder worden enkele keuzes besproken die zijn gemaakt bij het ranken van de concepten ten opzichte van elkaar:\\
\vspace{\baselineskip}

\textbf{Voorspelbaarheid concept.} Bij de voorspelbaarheid van het concept is de volgende ranking gegeven:

\begin{center}
 \begin{tabular}{| l | c | c | c | r | }
  \hline			
  x & De inklapper & De mantis car & De uitschuiver & De vierpoter \\
  Gegeven score: &  5 & 1 & 4 & 5 \\
  \hline  
 \end{tabular}
\end{center}

Bij het geven van deze scores is gekeken naar hoe het concept het talud neemt. Hoe simpeler de beweging is die het concept moet maken om over het talud te komen, hoe hoger de score bij dit criteria.\\
Daarom heeft de mantis car een lage score, het concept berust op het feit dat dit systeem zichzelf omhoog hijst door middel van een soort klauw. Deze beweging is ingewikkelder en daarom minder voorspelbaar dan een mechanisme dat als het ware over het talud stapt.\\
\vspace{\baselineskip}

\textbf{Prijs.} Bij het geven van deze scores is gekeken naar hoeveel dure onderdelen moeten worden aangeschaft per concept. De eventuele kostenposten zijn onderzocht op \cite{Conrad} en op \cite{zwaard_delft} .

Dit waren de kostenposten die uit het onderzoek naar boven kwamen: \\

\begin{itemize}
    \item Elektromotoren
    \item Arduino
    \item Hoeveelheid materiaal
    \item Accu
\end{itemize}

Hieruit blijkt dat elektra veel bepalend is voor de uiteindelijk kosten. Daarom zijn de scores voornamelijk gebaseerd op de hoeveelheid elektra en de kosten ervan. Dit zijn de gegeven scores: \\

\begin{center}
 \begin{tabular}{| l | c | c | c | r | }
  \hline			
  x & De inklapper & De mantis car & De uitschuiver & De vierpoter \\
  Gegeven score: & 4 & 5 & 2 & 4 \\
  \hline  
 \end{tabular}
\end{center}

Zoals te zien is heeft de Mantis Car een hoge score hier. Dit heeft te maken met het feit dat de mantis car gebruik maakt van maar 3 elektromotoren. In tegen stelling tot de concepten met inklapbare pootjes of schaar liften, hoeft dit concept geen pootjes in te klappen. Doordat de Mantis Car minder elektromotoren nodig heeft, heeft dit concept de hoogste score.\\ 
\vspace{\baselineskip}


\section{Keuze van Concept}
\label{se:keuze_van_concept}
Uit het onderzoek is gebleken dat op de eerste plaats de \textbf{Inklapper} is gekomen. Dit heeft te maken met de hoge scores die hij heeft gekregen bij het talud nemen en de voorspelbaarheid.\\

Dicht achter de inklapper kwam de vierpoter. In principe is een verschil van 4 punten niet doorslag gevend in de keuze van het concept. Toch is er voor de inklapper is gekozen door de aanwezigheid van een rood vlak voor de score tabel van de vierpoter. Dit rode vlak geeft aan dat de vierpoter onvoldoende scoort op het gewicht van het concept, hierdoor kan het concept wel op papier kansrijk zijn maar in praktijk is misschien het tegenover gestelde. Deze afweging heeft ervoor gezorgd dat er is gekozen voor de Inklapper.\\